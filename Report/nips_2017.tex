\documentclass{article}

% if you need to pass options to natbib, use, e.g.:
\PassOptionsToPackage{numbers, compress}{natbib}
% before loading nips_2017
%
% to avoid loading the natbib package, add option nonatbib:
% \usepackage[nonatbib]{nips_2017}

\usepackage{nips_2017}

% to compile a camera-ready version, add the [final] option, e.g.:
% \usepackage[final]{nips_2017}

\usepackage[utf8]{inputenc} % allow utf-8 input
\usepackage[T1]{fontenc}    % use 8-bit T1 fonts
\usepackage{hyperref}       % hyperlinks
\usepackage{url}            % simple URL typesetting
\usepackage{booktabs}       % professional-quality tables
\usepackage{amsfonts}       % blackboard math symbols
\usepackage{nicefrac}       % compact symbols for 1/2, etc.
\usepackage{microtype}      % microtypography
\usepackage{amsmath}

\title{Second Order Momentum}

\author{
  Vaibhav Sinha \\
  \texttt{cs15btech11034@iith.ac.in} \\
  \And
  Sukrut Rao \\
  \texttt{cs15btech11036@iith.ac.in} \\
}

\begin{document}

\maketitle

\begin{abstract}
  We study the effect of momentum on second order methods, both Newton's method as well as Quasi-Newton methods like LBFGS. We empirically determine the performance as compared to existing classes of methods.
\end{abstract}

\section{Project Objective}

To achieve faster convergence, second order methods were proposed over first order methods. Although second order methods have better theoretical bounds than first order methods and converge in very few steps, they are often not used in practice due to the cost of each step. To cope with these problems, two approaches have been proposed and well studied: (i) Adding momentum \cite{POLYAK19641,Sutskever} to first order methods to make the convergence speed comparable to second order methods, (ii) Quasi-Newton methods \citep{davidon,broyden1965class,nocedal1980updating} which avoid the costly computations involved in second order methods, with a small cost on the rate of convergence. In this project, we aim to combine both these approaches and attempt to verify if the classical second order methods as well as Quasi-Newton methods can be accelerated.

\section{Novelty and Proposed Methodology}

Momentum has been extensively studied on first order methods \citep{POLYAK19641,Sutskever}. However, there seems to be almost no work done on the effects of momentum on Newton’s method. Through this project we seek to address this issue.

Moreover, to accelerate second order methods, several Quasi-Newton methods have been proposed including BFGS \citep{broyden1965class}, which was subsequently improved for memory performance by LBFGS \citep{nocedal1980updating}. We also plan to check the performance of momentum on BFGS/LBFGS and the Cubic regularized Newton method \citep{nesterov2006cubic}.

\subsection{Brief ideas}

For first order methods the momentum equation is given as,
\begin{align*}
v^{+} &= \gamma v + (1-\gamma) \nabla f(x) \\
x^{+} &= x - tv^{+}
\end{align*}

In a similar manner, we propose the momentum equation for second order methods,

\begin{align*}
v^{+} &= \gamma v + (1-\gamma) \nabla^2 f(x)^{-1} \nabla f(x) \\
x^{+} &= x - tv^{+}
\end{align*}

Similarly, we will explore momentum for BFGS. We intend to study both the classical form of momentum \citep{POLYAK19641} as well as the one used in deep learning proposed by \citet{Sutskever}

\section{Planned Experiments}

We will compare the performance of first order methods, first order methods with momentum, and second order methods with our method. We plan to test on a range of benchmark problems collated by \citet{jamil2013literature}.

\section{Performance Metrics and Expected Results}

A well performing optimization method is characterized by a high rate of convergence over a wide range of problems. The performance of different techniques can be compared by finding the distance from the optimum after a set of fixed, predefined number of iterations.

In other words, we compute
\begin{equation}
|f(x^{(k)}) - f(x^*)|
\end{equation}

for fixed values of $k$ and a set of functions $f$ for each class of methods.

We expect to characterize the performance of using momentum with second order methods on a set of benchmark problems taken from \citep{jamil2013literature}, to enable a comparison with existing methods.

\bibliographystyle{unsrtnat}
\bibliography{references}

\end{document}
